La rapida crescita delle librerie accademiche e l'avvento di nuove tecnologie, che tutt'ora stanno caratterizzando il XXI secolo, ha scaturito la diffusione di enormi quantitivi di informazioni veicolate online \cite{2023arXiv230913761W}. Fenomeno di portata talmente elevata che ha sancito la nascita dei \textbf{Big Data}. \vspace{7pt} \\
Con Big Data si indicano genericamente raccolte estremamente ampie e variegate di dati, da cui è possibile ricavare nuove conoscenze attuabili in differenti contesti. \vspace{7pt} \\
I dati collezionati rientrano in due ampie categorie, suddivise in: \textbf{dati strutturati} e \textbf{dati non-strutturati}. I dati strutturati sono informazioni che si adattano agevolmente a schemi predefiniti, quali formati tabellari, contraddistinti da un insieme finito di features che ne descrivono le caratteristiche. Mentre, i dati non-strutturati simboleggiano tutte quelle informazioni che ad un primo impatto non possono essere costrette all'interno di un qualsiasi schema predefinito, esempi sono file audio, video, oppure documenti di testo di grandi dimensioni. Questi ultimi sono molto più facili da reperire, ma al tempo stesso potrebbero richiedere molteplici sforzi affinché siano convertiti in un formato idoneo, successivamente sfruttabili dai modelli di apprendimento automatico. 