Malgrado l'applicazione di estrattori di dati, alcuni PDF potrebbero non essere compatibili, da cui la potenziale assenza di informazioni associate. Pertanto, possono essere adotatti ulteriori metodi ausiliari, pur di estendere l'analisi condotta ad un bacino più vasto possibile di documenti, tra cui:
\begin{itemize}
    \renewcommand{\labelitemi}{-}
    \item \textbf{Librerie Python}. \\
    Python è uno dei linguaggi di programmazione con il numero maggiore di librerie dedicate alla lettura e manipolazione di PDF. La facilità di sviluppo di questi strumenti rappresenta una valida alternativa per tutti quei documenti incompatibili con estrattori di informazioni. A livello implementativo, l'impiego delle librerie si traduce in una banale operazione di lettura dei documenti: una volta specificato il percorso del file, è restituito un oggetto caratterizzato dall'insieme di tutti i parametri recuperati. Nonostante la varietà di opzioni disponibili, i risultati ottenuti dalle librerie tendono a essere molto simili. Tuttavia, spesso differiscono in base alla velocità di elaborazione e all'accuratezza delle informazioni estratte, differenze che potrebbero influenzare la scelta dello strumento più adatto a seconda delle esigenze implementative. Tra le librerie più utilizzate è possibile citare \textbf{PyPDF} \cite{pypdf2024}, \textbf{PyMuPDF} \cite{pymupdf} e \textbf{PDFPlumber} \cite{pdfplumber}, ognuna contraddistinta da propri punti di forza e di debolezza.
    \item \textbf{Optical Character Recognition (OCR)}. \\
    Qualora l'estrazione di dati sia formalizzata su una collezione di file PDF, occorre considerare l'eventualità che alcuni documenti non siano creati digitalmente, ma realizzati successivamente ad una scansione. A tal proposito, le tecniche di elaborazione descritte fino ad ora potrebbero non garantire gli esiti attesi. Tuttavia, è possibile adoperare meccanismi di Optical Character Recognition, in cui immagini di testo sono convertite in un formato leggibile dalla macchina. Anche in questo contesto, Python offre svariate soluzioni relative a OCR, come \textbf{PyTesseract}. Python-tesseract è un \textbf{wrapper} della libreria Tesseract-OCR sviluppata da Google, progettato per ricavare il contenuto testuale intrinseco a immagini. Per utilizzare il package, è necessario trasformare i PDF in possesso in immagini, procedimento che potrebbe aumentare i tempi di esecuzione. Inoltre, solitamente, vengono attuati diversi filtri volti a migliorare la precisione e la velocità di lettura, modificando la scala di colori dell'immagine: spesso è applicata una gradazione scura, in grado di mettere in risalto il contenuto rispetto allo sfondo.  
\end{itemize}