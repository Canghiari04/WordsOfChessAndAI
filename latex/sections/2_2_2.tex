La raccolta di metadati inerenti ad articoli scientifici richiede l'accesso a una vasta gamma di risorse informative. Tuttavia, in queste casistiche, occorre utilizzare repository attendibili, che garantiscano la massima accuratezza. Il contesto universitario si distingue per un'ampia varietà di sistemi dedicati al recupero di parametri relativi a pubblicazioni accademiche, noti come \textbf{sistemi di recupero di riferimenti}. \vspace{7pt} \\
Un sistema di recupero di riferimenti è un insieme di strumenti e metodi utilizzati per individuare, raccogliere e organizzare informazioni relative a fonti documentarie, come libri, paper, tesi e altri materiali. Questi sistemi facilitano la ricerca e la gestione delle fonti, garantendo coerenza e precisione nelle citazioni e nelle bibliografie. \vspace{7pt} \\
Nel panorama accademico, è possibile usufruire di un'elevata molteplicità di tali sistemi, ognuno dei quali offre differenti funzionalità volte a facilitare il processo di acquisizione delle risorse, come CrossRef, arXiv, piuttosto che OpenAlex. \vspace{7pt} \\
\textbf{CrossRef} è un'associazione indipendente per la costruzione di tecnologie condivise. La sua missione è quella di migliorare l'accesso alle pubblicazioni scientifiche, attraverso servizi basati su accordi collettivi con editori del settore accademico e professionale \cite{Howells01042006}. \vspace{7pt} \\
La progettazione dell'infrastruttura è architettata con uno scopo ben definito: facilitare la trascrizione delle citazioni all'interno delle pubblicazioni. Ciò avviene mediante l'impiego di \textbf{Digital Object Identifier}, estensione dell'acronimo \textbf{DOI}, identificativi univoci e persistenti di una qualsiasi risorsa digitale riconosciuta. Inoltre, offre numerosi servizi volti all'estrapolazione di metadati inerenti a lavori di ricerca, tra cui una solida \textbf{REST API}, in maniera tale da ottenere il probabile codice univoco una volta fornito  il titolo e la lista di autori del documento. \vspace{7pt} \\
\textbf{OpenAlex} è un catalogo bibliografico che raccoglie un'ampia gamma di risorse e offre una robusta REST API, gratuita e priva di autenticazione, per il recupero di metadati relativi a specifici paper \cite{openalex}. Va sottolineato che la maggior parte delle informazioni reperibili tramite OpenAlex provengono direttamente da CrossRef, il che potrebbe renderlo uno strumento marginale. Nonostante ciò, il suo ruolo è rafforzato dalla rilevazione del database \textbf{Microsoft Academic}, precedentemente riconosciuto come il principale archivio di pubblicazioni provenienti da conferenze.