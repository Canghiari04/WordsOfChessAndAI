Dalle proprietà del dataset dipendono i modelli di classificazione adoperabili. Se l'analisi condotta dovesse formalizzarsi su una collezione di documenti, da cui l'assenza di dati etichettati, allora sarà possibile usufruire solamente di algoritmi di tipo non-supervisionato. \vspace{7pt} \\
Al fine di categorizzare le informazioni in possesso, possono essere valutate differenti opportunità. Tendenzialmente, modelli non-supervisionati prevedono l'impiego di \textbf{cluster}, ovvero raggruppamenti di istanze simili all'interno di medesimi insiemi. Al tempo stesso, i \textbf{Large Language Models} rappresentano una valida alternativa, dati i continui miglioramenti dimostrati durante l'esecuzione di task in ambito NLP, anche se in questa circostanza è necessario definire a priori le classi che dovranno essere impiegate durante la fase di catalogazione dell'osservazioni. \vspace{7pt} \\
Nei paragrafi seguenti sono illustrate alcune delle classificazioni applicabili, in accordo con quanto anticipato, nel caso in cui il dataset scelto non abbia osservazioni etichettate.