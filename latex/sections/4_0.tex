Il passaggio finale del caso di studio prevede di classificare i metadati acquisiti durante la fase di estrazione (\ref{3.0}), mediante l'applicazione di specifiche tecniche di machine learning. L'impiego degli algoritmi di apprendimento automatico, descritti nel Paragrafo \ref{2.3}, avviene per soddisfare la richiesta di categorizzazione degli articoli dell'archivio. \vspace{7pt} \\
A causa dell'assenza di classi predefinite associabili alle osservazioni raccolte, è emersa l'esigenza di selezionare alcune \textbf{etichette}, affinché includessero temi discussi dai documenti della collezione. \vspace{7pt} \\
Complessivamente, i modelli di classificazione sono stati implementati secondo un approccio Zero-Shot, volto a riconoscere pattern ripetitivi all'interno dell'entità date in ingresso, basandosi esclusivamente sulle conoscenze in possesso. A tal proposito, sono stati adoperati modelli \textbf{pre-addestrati} e \textbf{non-supervisionati}, data la mancanza di dati annotati. \vspace{7pt}\\
I paragrafi successivi evidenzieranno la capacità marginale delle tecniche di machine learning durante la classificazione dei paper, nonostante sia stata condotta un'attività preliminare di estrazione dei dati incentrata sulla qualità delle informazioni piuttosto che sulla quantità disponibile.