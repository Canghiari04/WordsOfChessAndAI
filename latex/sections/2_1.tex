Il \textbf{Natural Language Processing}, estensione della sigla comunemente utilizzata \textbf{NLP}, è una branca dell'intelligenza artificiale che impiega un insieme di tecniche di \textbf{machine learning} per consentire alle macchine di comprendere e sfruttare il linguaggio umano. \vspace{7pt} \\
Il primo a introdurre il tema fu il matematico \textbf{Alan Turing}, all'interno del celebre articolo intitolato \textbf{Computing Machinery and Intelligence}, in cui lo scienziato propose quello che oggi è noto come \textbf{Turing test}: affinché un calcolatore possa essere ritenuto intelligente, deve dimostrare la capacità di generare e interpretare autonomamente il linguaggio naturale. Pur rappresentando uno stadio primordiale della letteratura, costituì il primo passo essenziale che ha reso possibile lo sviluppo di molteplici funzionalità legate all'AI, divenute attualmente una parte imprescendibile della quotidianità. \vspace{7pt} \\
NLP è diventata una tecnologia fondamentale in diversi settori, rivoluzionando il modo in cui le macchine e gli esseri umani interagiscono. Tuttavia, la sua applicazione non è banale, e spesso richiede l'implementazione di specifici passaggi pur di raggiungere il risultato attesso, i quali possono essere sintetizzati in:
\begin{itemize}
    \renewcommand{\labelitemi}{-}
    \item \textbf{Elaborazione e conversione dei dati}. \\
    L'elaborazione dei dati consiste nel recupero e trasformazione di informazioni testuali in un formato facilmente manipolabile e comprensibile da parte della macchina. Solitamente ad attività di estrapolazione di dati sono affiancate operazioni di preprocessing, le quali rappresentano un crocevia essenziale pur di ottenere dei risultati finali attendibili, dato che può influenzare profondamente l'accuratezza delle tecniche di classificazione; infatti, per questa ragione principale, spesso potrebbe essere preferito un approccio mirato a valorizzare la qualità delle osservazioni piuttosto che la quantità delle stesse. Inoltre, come è stato già ribadito, task compiute da modelli di machine learning non impiegano \textbf{dati non-strutturati}. Pertanto, è necessario convertite le informazioni testuali in possesso secondo una \textbf{rappresentazione numerica}, in modo tale che algoritmi di apprendimento automatico siano in grado di analizzare e interpretare le variabili in ingresso.
    \item \textbf{Analisi del testo}. \\
    Analizzare gli elementi a disposizione comporta all'estrazione di domini significativi. Un esempio è dato dalla \textbf{topic extraction}, in cui sono identificati argomenti comuni ad un corpus di documenti.
    \item \textbf{Addestramento dei modelli}. \\
    I dati elaborati vengono utilizzati per addestrare modelli di machine learning, ritenuti oggi fondamenta su cui si basano differenti tecniche di Natural Language Processing, affinché il linguaggio naturale possa essere interpretato dai calcolatori. Questi modelli, durante il training, riconoscono pattern ripetitivi e e le loro correlazioni, regolando i propri parametri per minimizzare i potenziali errori e migliorare le proprie prestazioni.
\end{itemize}