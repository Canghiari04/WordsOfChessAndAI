I risultati ottenuti al termine dell'analisi hanno evidenziato numerose problematiche relative alla classificazione delle osservazioni recuperate secondo liste predefinite di etichette. Sebbene siano stati adeguati differenti accorgimenti durante la fase di estrazione di dati, i modelli di linguaggio naturale e di topic extraction non hanno mai restituito esiti considerati soddisfacenti. \vspace{7pt} \\
Durante lo sviluppo degli esperimenti è emerso un punto di debolezza comune per ogni algoritmo di apprendimento automatico utilizzato. In particolare, ciascuna tecnica implementata ha mostrato difficoltà nell'elaborazione di dati testuali complessi, soprattutto qualora il significato semantico delle informazioni contenute non era facilmente interpretabile. \vspace{7pt} \\
I modelli di linguaggio, infatti, pur essendo in grado di manipolare elevate quantità di testo, hanno mostrato significative vulnerabilità in circostanze in cui occorresse comprendere profondamente il tema trattato dal contenuto testuale. Inoltre, le annotazioni peggiori sono stati ottenute nei casi in cui la lista di etichette impiegata fosse caratterizzata da elementi concettualmente molto simili tra loro, in cui la capacità discriminativa dei modelli è diminuita drasticamente. \vspace{7pt} \\
Viceversa, gli strumenti attuati durante la fase di estrazione dei dati dall'insieme di articoli scientifici presi in esame, hanno garantito esiti più consistenti e affidabili, seppur con qualche difficoltà. GROBID e librerie Python per la lettura di file PDF, non hanno dimostrato la stessa efficacia nel manipolare il contenuto di file successivi ad una scansione rispetto a documenti creati digitalmente, scoraggiando l'estrapolazione di ulteriori dati. \vspace{7pt} \\
Nonostante gli obiettivi dichiarati dal caso di studio non siano stati completamente rispettati, le criticità emerse hanno evidenziato ulteriori opportunità di sviluppo. Ad esempio, l'incapacità degli strumenti attuali di gestire file scansionati suggerisce la possibilità di integrare meccanismi di Optical Character Recognition, in maniera tale che possano essere recuperate informazioni aggiuntive precedentemente impossibili da ricavare. In aggiunta, potrebbe essere considerata l'adozione di modelli fine-tuned su dataset specifici, concordi con le stesse tematiche trattate dalla collezione di pubblicazioni accademiche considerate, contribuendo a un miglioramento della comprensione e dell'interpretazione semantica. \vspace{7pt} \\
In sintesi, sebbene i risultati non abbiano pienamente soddisfatto le aspettative iniziali, le lacune individuate offrono una base solida per sviluppi futuri. Questa prospettiva permette non solo di colmare le attuali carenze, ma apre anche a nuove opportunità di ricerca e innovazione, rendendo la tesi presentata un punto di partenza piuttosto che un punto di arrivo.