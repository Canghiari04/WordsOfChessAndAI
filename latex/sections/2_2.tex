L'estrapolazione dei dati corrisponde al processo di recupero di informazioni provenienti da fonti eterogenee, spesso non-strutturate, in maniera tale che possano essere convertite in un formato idoneo e facilmente gestibile. Un esempio concreto di quanto riportato, potrebbe avvenire per realtà aziendali che desiderino analizzare la percezione dei propri prodotti da parte dei clienti, basandosi sulle recensioni online. \vspace{7pt} \\
Si consideri un'azienda intenzionata di migliorare i suoi prodotti e a capire quale sia il loro indice di gradimento. In tale ambito, le recensioni costituiscono una fonte ricca di informazioni, ma sono tipicamente espresse in linguaggio naturale, quindi di natura non-strutturata. Per questo motivo, è fondamentale raccogliere un ampio volume di recensioni, riorganizzando i dati contenuti rispetto ad una struttura tabellare, in modo tale che siano agevolmente interpretabili. \vspace{7pt} \\
Il caso di studio della tesi segue sommariamente il titolo esemplificativo. Uno degli obiettivi dell'analisi è ricavare metadati dalla lista di articoli accademici in possesso, sfruttando tecniche di estrazione di informazioni da documenti e sistemi per il recupero dei riferimenti bibliografici. In questo modo, sarà possibile costruire un dataset che potrà essere utilizzato nella fase successiva di classificazione. \vspace{7pt} \\