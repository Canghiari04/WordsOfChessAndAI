La preponderanza di PDF ha preteso l'utilizzo di strumenti appositi per la conversione delle risorse in un formato facilmente manipolabile. Nonostante Python sia caratterizzato da numerosi package dedicati alla lettura di file PDF, in alcune circostanze i risultati non sono attendibili. Infatti, la maggior parte delle librerie, alcune di esse definite nel Paragrafo \ref{2.2.3}, hanno riscontrato stesse problematiche relative all'estrazione di dati, tra cui: mancato o erroneo riconoscimento delle sezioni qualora il contenuto dell'articolo fosse disposto in due colonne, incompletezza dei metadati e assenza di informazioni per documenti ottenuti tramite scansione. \vspace{7pt} \\
Da queste difficoltà è emersa la necessità di ripiegare su ulteriori sistemi. Sebbene le tecniche precedenti garantissero una maggiore velocità di esecuzione, GROBID ha estrapolato il contenuto della maggior parte delle risorse presenti nell'archivio. \vspace{7pt} \\
La scelta implementativa è ricaduta sulla combinazione complessiva degli strumenti citati, in modo tale che l'analisi fosse realizzata su uno spettro più ampio possibile. Medesima decisione è stata adeguata per il recupero dei riferimenti bibliografici. CrossRef ha fornito il numero maggiore di riferimenti, ma alcuni articoli risultano assenti nella sua infrastruttura. Per questa ragione, sono state integrate ulteriori REST API specializzate per il recupero di fonti bibliografiche.