La classificazione testuale è un problema centrale in ambito Natural Language Processing, con applicazioni che variano dall'etichettatura fino all'estrapolazione di argomenti ricorrenti da un insieme di dati. Il lavoro di questa tesi esplora le tecniche di machine learning applicate per annotare il contenuto di molteplici articoli scientifici, focalizzati sull'evoluzione del rapporto tra intelligenza artificiale e computer chess. L'obiettivo del caso di studio è la costruzione di un sistema capace di estrarre informazioni testuali da un archivio di file PDF e la valutazione dell'efficacia di modelli di apprendimento automatico, pre-addestrati e non-supervisionati, nella classificazione dei contenuti ricavati dalla medesima collezione di documenti. Successivamente ad attività di estrazione e preprocessing dei dati, poi racchiusi all'interno di un dataset, sono utilizzati modelli di linguaggio naturale e di topic extraction per tentare di classificare le osservazioni estrapolate secondo delle liste predefinite di categorie, seguendo un approccio Zero-Shot. L'analisi dei risultati finali ottenuti evidenzieranno le difficoltà di ciascun metodo impiegato, offrendo una visione per futuri sviluppi nella classificazione testuale di documenti accademici.