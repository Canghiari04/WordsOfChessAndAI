Sebbene GROBID abbia garantito ottimi rendimenti, una porzione di risorse dell'archivio non fu compatibile con il servizio, data la conversione in file XML/TEI carenti di contenuti. Per non perdere informazioni preziose si è deciso di impiegare ulteriori librerie Python dedicate alla lettura di file PDF, nella speranza di raccimolare qualche dato aggiuntivo. \vspace{7pt} \\
A partire dalla lista di oggetti di tipo Metadata, introdotta nel Paragrafo \ref{3.2.1}, sono stati individuati gli articoli privi di parametri. I documenti circoscritti sono stati assegnati al modulo \textbf{PDFReader} del package \textbf{PyPDF}, in maniera tale che fosse possibile accedere alle proprietà dell'istanza successivamente alla lettura dei file.
\begin{lstlisting}[language=python, caption=Lettura dei documenti PDF tramite la libreria PyPDF]
pattern = (r"(^.*)?.grobid")

def retrieve_missing_metadata(list_metadata: List[Metadata]):
    for metadata in list_metadata:
        if metadata.title is None or len(metadata.author) == 0:
            pdf_path = re.search(
                pattern,
                metadata.path
            ).group(1) + ".pdf"

            try:
                _pdfreader = PdfReader(input_dir + pdf_path).metadata
            except Exception:
                pass
             
            title = _pdfreader.title or None
            author = _pdfreader.author or []

            _.title = title
            _.path = pdf_path
            _.author = define_authors(author)\end{lstlisting}
Nonostante siano state valutate differenti librerie, è stata implementato un singolo package. La scelta è stata dettata dal fatto che gli esiti ottenuti dalle molteplici librerie non variavano significativamente, distinguendosi principalmente per la velocità di esecuzione. Pertanto, si è optato per una soluzione che garantisse un miglior compromesso tra facilità di sviluppo e prestazioni.