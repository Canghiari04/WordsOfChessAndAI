D'ora in poi verranno messi in pratica i concetti esposti nel Capitolo \ref{2.0}, concentrando il caso di studio sugli esiti ottenuti dallo sviluppo. \vspace{7pt} \\
Prima di applicare gli algoritmi di classificazione, sono stati creati due dataset contenenti alcuni parametri relativi a un archivio di articoli accademici. Inizialmente i documenti sono stati elaborati per estrarne il contenuto, grazie all'impiego di GROBID e di appositi package (\ref{2.2}). \vspace{7pt} \\
I dati estrapolati sono stati poi inseriti in un lista di oggetti di tipo \textbf{Metadata} (\ref{lst:Metadata}), in modo tale che sezioni degli articoli fossero facilmente manipolabili. Attività che si è rilevata piuttosto utile durante la fase di acquisizione di DOI, ottenuti attraverso le REST API fornite dai principali sistemi di recupero di riferimenti bibliografici. \vspace{7pt} \\
Infine, a partire dalla struttura dati ottenuta, sono stati implementati due dataset, mediante la libreria \textbf{Pandas} di Python, racchiudendo l'insieme di tutte le osservazioni recuperate. La suddivisione in due dataframe avviene per rispondere alle esigenze architetturali dei modelli di apprendimento automatico adoperati.